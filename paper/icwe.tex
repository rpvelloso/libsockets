\documentclass[runningheads]{llncs}

%\usepackage{booktabs} % For formal tables

% Copyright
%\setcopyright{none}
%\setcopyright{acmcopyright}
%\setcopyright{acmlicensed}
%\setcopyright{rightsretained}
%\setcopyright{usgov}
%\setcopyright{usgovmixed}
%\setcopyright{cagov}
%\setcopyright{cagovmixed}

%\usepackage[utf8]{inputenc}
\usepackage{graphicx}
%\usepackage{algorithmicx}
%\usepackage[Algorithm,ruled]{algorithm}
%\usepackage[noend]{algpseudocode}
%\usepackage{url}
%\usepackage{balance}
%\usepackage{listings}
%\usepackage{enumerate}
\usepackage{color}
%\usepackage{microtype}
%\usepackage[medium,compact]{titlesec}

\usepackage{pifont}% http://ctan.org/pkg/pifont
\newcommand{\cmark}{\ding{51}}%
\newcommand{\xmark}{\ding{55}}%

\hyphenation{spec-trum}

\begin{document}

\title{Web Page Structured Content Detection using Supervised Machine Learning}

%\author{Roberto Panerai Velloso}
%\affiliation{%
%  \institution{Universidade Federal de Santa Catarina}
%  \city{Florianópolis} 
%  \state{Santa Catarina} 
%  \country{Brazil}
%}
%\email{rvelloso@gmail.com}
%
%\author{Carina F. Dorneles}
%\affiliation{%
%  \institution{Universidade Federal de Santa Catarina}
%  \city{Florianópolis} 
%  \state{Santa Catarina} 
%  \country{Brazil}
%}
%\email{dorneles@inf.ufsc.br}

%\author{Author A}
%\affiliation{%
%  \institution{University}
%  \city{City} 
%  \state{State} 
%  \country{Country}
%}
%\email{e-mail@domain.com}

%\author{Author B}
%\affiliation{%
%  \institution{University}
%  \city{City} 
%  \state{State} 
%  \country{Country}
%}

\author{Roberto Panerai Velloso\inst{1} \and
Carina F. Dorneles\inst{1}}

\institute{Universidade Federal de Santa Catarina, Florianopolis/SC
Brasil\email{rvelloso@gmail.com@inf.ufsc.br} \and
\email{dorneles@inf.ufsc.br}}


%\email{e-mail@domain.com}

\begin{abstract}
    In this paper we present a comparative study using several supervised 
    machine learning techniques, including homogeneous and heterogeneous
    ensembles, to solve the problem of classifying content and noise in web
    pages. We specifically tackle the problem of detecting content in 
    semi-structured data (e.g., e-commerce 
    search results) under two different
    settings: a controlled environment with only structured content documents
    and; an open environment where the web page being processed may or may
    not have structured content.
    The features are obtained from a preexisting and publicly available extraction 
    technique that processes web pages as a sequence of tag paths, thus 
    the features are extracted from these sequences instead of the DOM 
    tree. Besides comparing the performance between different models 
    we have also conducted extensive feature selection/combination 
    experiments. We have achieved an average F-score of about
    93\% in a controlled setting and 91\% in an open setting.
\keywords{web mining, content detection, noise removal, record extraction,
structure detection, information retrieval}
\end{abstract}

\input{icwebody}

\bibliographystyle{splncs04}
\bibliography{refs} 

\end{document}
