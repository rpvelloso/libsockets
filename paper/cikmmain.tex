\documentclass[sigconf]{acmart}

\usepackage{booktabs} % For formal tables


% Copyright
%\setcopyright{none}
%\setcopyright{acmcopyright}
%\setcopyright{acmlicensed}
\setcopyright{rightsretained}
%\setcopyright{usgov}
%\setcopyright{usgovmixed}
%\setcopyright{cagov}
%\setcopyright{cagovmixed}

\usepackage[utf8]{inputenc}
\usepackage{graphicx}
\usepackage{algorithmicx}
\usepackage[Algorithm,ruled]{algorithm}
\usepackage[noend]{algpseudocode}
\usepackage{url}
\usepackage{balance}
\usepackage{listings}
\usepackage{enumerate}
\usepackage{color}
\usepackage{microtype}
\usepackage[medium,compact]{titlesec}

\hyphenation{spec-trum}

% DOI
\acmDOI{10.475/123_4}

% ISBN
\acmISBN{123-4567-24-567/08/06}

%Conference
\acmConference[Singapore 2017]{CIKM International Conference on Information and Knowledge Management}{November 2017}{Singapore} 
\acmYear{2017}
\copyrightyear{2017}

\acmPrice{15.00}


\begin{document}
\title{Extracting Records from the Web Using a Signal Processing Approach}

\author{First Author}
\affiliation{%
  \institution{Institution}
  \city{City} 
  \state{State} 
  \country{Country}
}
\email{email@host.com}

\author{Second Author}
\affiliation{%
  \institution{Institution}
  \city{City} 
  \state{State} 
  \country{Country}
}
\email{email@host.com}

\iffalse
\author{Roberto Panerai Velloso}
\affiliation{%
  \institution{Universidade Federal de Santa Catarina}
  \city{Florianópolis} 
  \state{Santa Catarina} 
  \country{Brazil}
}
\email{rvelloso@gmail.com}

\author{Carina F. Dorneles}
\affiliation{%
  \institution{Universidade Federal de Santa Catarina}
  \city{Florianópolis} 
  \state{Santa Catarina} 
  \country{Brazil}
}
\email{dorneles@inf.ufsc.br}
\fi

\begin{abstract}
Extracting records from web pages enables a number of important
applications and has immense value due to the amount and diversity of
available information that can be extracted. This problem, although vastly
studied, remains open because it is not a trivial one. Due to the scale of data,
a feasible approach must be both automatic and efficient (and of course
effective). We present here a novel approach, fully automatic and
computationally efficient, using signal processing techniques to detect
regularities and patterns in the structure of web pages. Our approach segments
the web page, detects the data regions within it, identifies the records
boundaries and aligns the records. Results show high f-score and
linearithmic time complexity behaviour.
\end{abstract}

%
% The code below should be generated by the tool at
% http://dl.acm.org/ccs.cfm
% Please copy and paste the code instead of the example below. 
%
\begin{CCSXML}
<ccs2012>
<concept>
<concept_id>10002951.10003260.10003277.10003279</concept_id>
<concept_desc>Information systems~Data extraction and integration</concept_desc>
<concept_significance>500</concept_significance>
</concept>
<concept>
<concept_id>10002951.10003317.10003318.10003319</concept_id>
<concept_desc>Information systems~Document structure</concept_desc>
<concept_significance>500</concept_significance>
</concept>
<concept>
<concept_id>10002951.10003317.10003347.10003352</concept_id>
<concept_desc>Information systems~Information extraction</concept_desc>
<concept_significance>500</concept_significance>
</concept>
<concept>
<concept_id>10002951.10003317.10003318.10003321</concept_id>
<concept_desc>Information systems~Content analysis and feature selection</concept_desc>
<concept_significance>300</concept_significance>
</concept>
</ccs2012>
\end{CCSXML}

\ccsdesc[500]{Information systems~Data extraction and integration}
\ccsdesc[500]{Information systems~Document structure}
\ccsdesc[500]{Information systems~Information extraction}
\ccsdesc[300]{Information systems~Content analysis and feature selection}

\keywords{web mining, record extraction, structure detection, information retrieval, record alignment}

\maketitle

\input{cikmbody}

\bibliographystyle{ACM-Reference-Format}
\bibliography{refs} 

\end{document}
