\documentclass{beamer}
\usepackage[brazil]{babel}
\usepackage[utf8]{inputenc}
\usepackage[T1]{fontenc}
\usepackage{graphicx}
\usepackage{array}
\usepackage{listings}
\usepackage{url}
\usepackage{multicol}


%%%%%%%%%%%%%%%%%%%%%%%%%%%%%%%%%%%%%%%%%%%%%%%%%%%%%%%%%%%%%%%%%%%%%%%%%%%%%%%%
%
%  \zoombox[box line width]{contents}
%
%  optimized version for beamer: in full screen, zoom boxes are centred
%  in the viewer; useable with any documenclass
%
%%%%%%%%%%%%%%%%%%%%%%%%%%%%%%%%%%%%%%%%%%%%%%%%%%%%%%%%%%%%%%%%%%%%%%%%%%%%%%%%
\makeatletter
\newsavebox\zb@x
\newcounter{z@@m}
\usepackage{calc}
\newdimen\B@r\newdimen\P@r
\newdimen\@zw\newdimen\@zh\newdimen\@zd

\newcommand{\zoombox}[2][0]{%
  \leavevmode%
  \sbox\zb@x{#2}%
  \setlength\B@r{1pt*\ratio{\wd\zb@x}{\ht\zb@x+\dp\zb@x}}%
  \setlength\P@r{1pt*\ratio{\paperwidth}{\paperheight}}%
  \ifdim\B@r>\P@r\relax%
    \setlength\@zw{\wd\zb@x}\setlength\@zh{\@zw*\ratio{\paperheight}{\paperwidth}}%
    \setlength\@zd{(\@zh-\ht\zb@x-\dp\zb@x)*\real{0.5}+\dp\zb@x}%
    \setlength\@zh{\@zh-\@zd}%
  \else%
    \setlength\@zh{\ht\zb@x+\dp\zb@x}%
    \setlength\@zw{\@zh*\ratio{\paperwidth}{\paperheight}}%
    \setlength\@zh{\ht\zb@x}\setlength\@zd{\dp\zb@x}%
  \fi%
  \makebox[0pt][l]{\makebox[\wd\zb@x][c]{\makebox[\@zw][l]{%
    \pdfdest name {zbfs\thez@@m} fitr
      width  \@zw\space
      height \@zh\space
      depth  \@zd\space
  }}}%
  \pdfdest name {zb\thez@@m} fitr
    width  \wd\zb@x\space
    height \ht\zb@x\space
    depth  \dp\zb@x\space
  \immediate\pdfannot 
    width  \wd\zb@x\space
    height \ht\zb@x\space
    depth  \dp\zb@x\space
  {%
    /Subtype/Link/H/N
    /Border [0 0 #1 [1 2]]
    /A <<
      /S/JavaScript
      /JS (
        if(typeof(zoomed)=='undefined'||!zoomed){
          var lastView=this.viewState;
          if(app.fs.isFullScreen) this.gotoNamedDest('zbfs\thez@@m');
          else this.gotoNamedDest('zb\thez@@m');
          zoomed=true;
        }else{
          this.viewState=lastView;
          zoomed=false;
        }
      )
    >>
  }%
  \usebox{\zb@x}%
  \stepcounter{z@@m}%
} 
\makeatother
%%%%%%%%%%%%%%%%%%%%%%%%%%%%%%%%%%%%%%%%%%%%%%%%%%%%%%%%%%%%%%%%%%%%%%%%%%%%%%%


\usetheme{Laughlin}

\begin{document}
\title{Extracting Records from the Web Using a Signal Processing Approach}
%\author[Roberto Panerai Velloso]{Orientando: Roberto Panerai Velloso \\
%Orientadora: Carina F. Dorneles \\ \{rvelloso, dorneles\}@gmail.com }
\author[Roberto Panerai Velloso]{Roberto Panerai Velloso, Carina F. Dorneles \\ 
\{rvelloso, dorneles\}@gmail.com}
%\date{\today}
\date{}
%\institute{Universidade Federal de Santa Catarina}
\institute{
%UFSC - Universidade Federal de Santa Catarina \\ PPGCC - Programa de
% Pós-Graduação em Ciência da Computação
\begin{figure}[H]
  \label{fig:logo1}
    %\includegraphics[scale=0.50]{img/brasao_ufsc_80.png}
    %\includegraphics[scale=0.30]{img/brasao_418.png}
    \includegraphics[scale=0.30]{img/cikm.png}
\end{figure}
%\begin{figure}[H]
%  \label{fig:logo2}
%    \includegraphics[scale=0.30]{brasao_ufsc_80.png}
%\end{figure}
}

\frame{
\titlepage
} 

\frame{\frametitle{Table of Contents}
\begin{small}
\begin{multicols}{2}
  \tableofcontents
\end{multicols}
\end{small}
}

\section{Introduction}
%\frame{\tableofcontents[currentsection]}

\subsection{Context}
\frame{\frametitle{Context}
\begin{itemize}
\item History
\begin{itemize}
  \item about 15+ years of research on the subject;
  \item still an open problem (it is a hard one, indeed).
\end{itemize}
\item Extract structured data (\textit{i.e.}, relational) from semi-structured
sources.

\item Data is available for human consumption, not for machines.

\item Motivation
\begin{itemize}
  \item run (no)SQL queries over the data;
  \item assemble domain specific databases;
  \item enrich other applications with structured data.
\end{itemize}
\end{itemize}
}

\subsection{Problema}
\frame{\frametitle{Challenges}
\begin{itemize} 
	\item extraction from semi-structured sources;
	\item make use of syntactical information (our scope);
	\item scalability (large amount of data to be processed):
	\begin{itemize}
	  \item level of supervision;
	  \item generality (diversity of data);
	  \item precision vs recall.
\end{itemize}
\end{itemize}
}

\subsection{Objetivo}
\frame{\frametitle{Objetivo Geral} 
O objetivo geral desta pesquisa é desenvolver um método automático (não
supervisionado) de extração de dados estruturados a partir de informação
semiestruturada contida em páginas \textit{web}.
}

\frame{\frametitle{Objetivos específicos}
\begin{itemize}
\item desenvolver um método de \textbf{segmentação} de um documento HTML que
identifique as regiões da página com conteúdo semiestruturado;
\item desenvolver um método de \textbf{filtragem} que identifique quais das
regiões semiestruturadas possuem conteúdo e quais possuem ruído (menus, \textit{template}, anúncios,
etc.);
\item desenvolver um método de estruturação da informação semiestruturada
identificada (\textit{e.g.} \textbf{alinhamento} de registros);
\item desenvolver um \textbf{\textit{framework}} de extração estruturada;
\end{itemize}
}

\subsection{Contribuições}
\frame{\frametitle{Contribuições}
\begin{itemize}
  \item uma nova forma de ver a estrutura das páginas web;
  \item uma nova técnica de segmentação de documentos HTML;
  \item um novo critério para determinar se um segmento (\textit{i.e.}, região)
  é estruturado ou não, antes de extraí-lo efetivamente;
  \item uma técnica de extração com eficiência computacional superior ao estado
  da arte (linear);
  \item uma nova técnica de alinhamento de registros, específica para o
  problema de extração estruturada da web;
  \item desenvolvimento de um \textit{framework} de extração estruturada
  (auxilia nas comparações e elimina variabilidade dos testes).
\end{itemize}
}

\section{Trabalhos Relacionados} 
\frame{\frametitle{Trabalhos Relacionados}
\begin{itemize}
\item Poucas implementações disponíveis;
\item Alguns datasets disponíveis; 
\item Três principais abordagens identificadas:
\begin{itemize}
  \item extração de tabelas e listas;
  \item extração a partir de uma única página;
  \item extração a partir de múltiplas páginas.
\end{itemize}
\end{itemize}
}

\subsection{Tabelas e Listas}
\frame{\frametitle{Extração de Tabelas e Listas}
Abordagens fazem suposições \textbf{fortes} a respeito da entrada.
\begin{itemize}
  \item WebTables Google/ACSDb (\texttt{<table>}) (2008);
  \item WebTables M\$/Probase (\texttt{<table>}) (2012);
  \item ListExtract (\texttt{<ol/ul/etc.>}) (2009);
  \item Tegra (ListExtract da M\$) (\texttt{<ol/ul/etc.>}) (2015);
  \item \textit{Top-k pages} (\textit{template} fixo) (2013).
\end{itemize}
}

\subsection{Única Página}
\frame{\frametitle{Extração a partir de uma única página}
Buscam padrões no HTML/DOM para encontrar os registros. Ao menos dois registros
devem existir para ser possível a extração.
Algumas abordagens ``gerais'':
\begin{itemize}
  \item Abordagens não visuais: MDR, NET, TPC (2003 - 2009);
  \item Abordagens visuais: DEPTA, ViPER, ClustVX, AutoRM, (CHU et al., 2015b)
  (2005 - 2015);
\end{itemize}
}

\subsection{Múltiplas Páginas}
\frame{\frametitle{Extração a partir de múltiplas páginas}
Utilizam uma ``amostra'' do \textit{template} do \textit{site} para treinamento.
São capazes de extrair conteúdo de páginas com apenas um registro.
(\textit{i.e.}, páginas de detalhe).
\begin{itemize}
  \item RoadRunner (2001);
  \item ExAlg (2003);
  \item FiVaTech (2010).
\end{itemize} 
}

\section{Proposta}

\subsection{Características}
\frame{\frametitle{Características}
\begin{itemize}
  \item Independente de domínio;
  \item Independente de características da linguagem HTML; 
  \item Não dependa de bases de dados e/ou definições \textit{a priori}.
  \item Totalmente automática;
  \item Necessite de apenas uma página (ao invés de um conjunto de páginas);
  \item Não dependa de treinamento;
  \item Computacionalmente eficiente.
\end{itemize}
}

\subsection{Suposições}
\frame{\frametitle{Suposições}
\begin{enumerate}
\item \textbf{Segmentação}: diferentes
regiões de uma página HTML são formatadas de maneiras diferentes, justamente
para que haja distinção entre elas, portanto serão formadas por conjuntos de \textit{tag
paths} diferentes, caso contrário não haveria diferença visual
significativa entre as regiões de uma página;
\item \textbf{Registros}: regiões com conteúdo semiestruturado são compostas por
conjuntos de registros contíguos e com estrutura semelhantes, 
descritos por sequências de \textit{tag paths} semelhantes sendo, portanto, regiões cíclicas.
\end{enumerate} 
}


\subsection{Diagrama da abordagem}
\frame{\frametitle{Diagrama da abordagem}
\begin{figure}[H]
  \centering
    \zoombox{\includegraphics[width=1.00\textwidth]{img/proposal.jpg}}
\end{figure}
}

\subsection{Tag Paths}
\frame{\frametitle{$Tag$ $paths$}
\begin{figure}[H]
  \caption{Construção da sequência de $tag$ $paths$ (TPS) a partir do HTML.}
  \centering
    \zoombox{\includegraphics[clip, trim={1.1cm 1.9cm 1.5cm 1.65cm},
    width=1.00\textwidth]{img/tree2seq.pdf}}
\end{figure}
}
\frame{\frametitle{$Tag$ $paths$}
\begin{figure}[H]
  \caption{Construção da sequência de $tag$ $paths$ (TPS) a partir do HTML.}
  \centering
    \zoombox{\includegraphics[width=\textwidth]{img/creat-tps-pt.jpg}}
\end{figure}
}

\frame{\frametitle{$Tag$ $paths$}
\begin{figure}[H]
  \caption{Sequência de $tag$ $paths$ (TPS).}
  \centering
    \zoombox{\includegraphics[width=\textwidth]{img/tps-pt.jpg}}
\end{figure}
}

\subsection{Detecção das Regiões Semiestruturadas}
\frame{\frametitle{Contorno superior}
\begin{figure}[H]
  \caption{Contorno superior da TPS.}
  \centering
    \zoombox{\includegraphics[trim={55 200 40
     210},clip,scale=0.45]{img/contour-pt.pdf}}
\end{figure}
}
\frame{\frametitle{Derivada do Contorno}
\begin{figure}[H]
  \caption{Derivada do contorno superior da TPS.}
  \centering
    \zoombox{\includegraphics[trim={55 200 40
     210},clip,scale=0.45]{img/derivada-pt.pdf}}
\end{figure}
}
\subsection{Filtragem das Regiões}
\frame{\frametitle{Regressão Linear - Identificação da Estrutura}
\begin{figure}[H]
  \caption{Regressão linear da região.}
  \centering
    \zoombox{\includegraphics[scale=0.25]{img/region-pt.jpg}}
\end{figure}
}

\frame{\frametitle{Clustering das Regiões - Identificação do Ruído}
\begin{itemize}
\item O \textit{score} da região utilizado para clusterização é diretamente
proporcional ao tamanho da subsequência e inversamente proporcional à distância
que se encontra do centro da sequência completa;
\item É utilizado ``kmeans ótimo de uma dimensão'', forçando a criação de dois
\textit{clusters};
\item O \textit{cluster} com maior centro é considerado conteúdo e o outro
\textit{cluster} (com menor centro) é descartado.
\end{itemize}
}

\subsection{Subdivisão em Registros}
\frame{\frametitle{Detecção do tamanho e quantidade de registros}
\begin{figure}[H]
  \caption{a) TPS e regressão linear da região; b) PSD.}
  \centering
    \zoombox{\includegraphics[clip, trim={2cm 6cm 0.5cm 6.8cm},
    width=0.8\textwidth]{img/fftreg.pdf}}
\end{figure}
} \frame{\frametitle{Detecção do tamanho e quantidade de registros - reduzindo a
complexidade computacional} Transformada de Fourier possui complexidade
$O(nlogn)$ e calcula todo o espectro do sinal. Nesta aplicação, especificamente,
apenas uma faixa do espectro é necessária. Uma alternativa, para reduzir a
complexidade, é o algoritmo de Goertzel que calcula apenas um coeficiente em
tempo linear. Outra medida (alternativa ao Z-Score) deve ser encontrada para
avaliar se um coeficiente é significativo ou não (relação de Parseval). }

\subsection{Alinhamento dos Registros}
\frame{\frametitle{Center Star}
Como o alinhamento ótimo de múltiplas sequências é \textit{NP-Hard},
alternativas aproximadas, com complexidade polinomial, devem ser empregadas.

\begin{itemize}
\item Características do algoritmo \textit{Center Star}:
\begin{itemize}
  \item Não encontra necessariamente a solução ótima;
  \item Tem complexidade computacional polinomial $O(k^2\cdot n^2)$ (porém
  elevada);
  \item Garante um erro máximo com relação à solução ótima.
\end{itemize}
\end{itemize}
}

\frame{\frametitle{Solução de alinhamento específica para o problema}
As restrições do problema geral de alinhamento de múltiplas sequências não
precisam ser aplicadas, obrigatoriamente, ao problema específico de alinhamento
de múltiplos registros. Por exemplo, a ordem dos campos pode ser flexibilizada.
Soluções mais eficientes podem ser elaboradas relaxando o problema.
}

\section{Resultados}
\subsection{Precision, Recall \& F-Score}

\frame{\frametitle{Precision, Recall \& F-Score}
Resultados obtidos com dataset próprio contendo $1466$ registros de $46$
websites, de diversos domínios, das maiores empresas da internet:
\url{https://en.wikipedia.org/wiki/List_of_largest_Internet_companies}.

\begin{table}[h]
\centering
\begin{tabular}
{| c| c| c|}\hline
	Precision	& Recall	& F-Score\\ \hline
	98.95\% & 89.77\% & 94.13\% \\ \hline
\end{tabular}
\end{table}

Comparativo com outras técnicas estado da arte.
\begin{table}[h]
%\begin{small}
\begin{tabular}
{|c| c| c| c|}\hline
	& Precision	& Recall	& F-Score\\ \hline
MDR &	59,80\%	& 61,80\%	& 60,78\%\\ \hline
TPC	& 90,40\%	& 93,10\%	& 91,73\%\\ \hline
ClustVX &	99.81\% & 99.52\% & 99.66\%\\ \hline
Ours &	92,02\%	& 94,11\%	& 93,05\% \\ \hline
\end{tabular}
%\end{small}
%}
\end{table}


}

\subsection{Tempo de execução}
\frame{\frametitle{Tempo de execução}
\begin{figure}[h]
  \centering
     \includegraphics[clip, trim={2.0cm 7.0cm 2.0cm 7.3cm}, width=1.0\textwidth
     ]{img/runtime.pdf}
  \label{fig:runtime}
\end{figure}

}
%\frame{\frametitle{Tempo de execução}
%\begin{figure}[h]
%  \centering
%     \includegraphics[width=1.0\textwidth]{img/goertzel.png}
%  \label{fig:runtime}
%\end{figure}

%}

\section{Cronograma}
\frame{\frametitle{Cronograma da pesquisa}
\begin{enumerate}
  \item \textbf{levantamento bibliográfico}: leitura de publicações para
  levantamento do estado da arte \textbf{(até 12/2018)};
  \item \textbf{realização de experimentos}: realização de mais testes e
  experimentos com a abordagem proposta utilizando \textit{datasets} disponíveis publicamente
  para comparação com as demais abordagens existentes \textbf{(até 06/2018)};
  \item \textbf{publicação dos artigos}: escrita e publicação de artigos em
  congressos e periódicos \textbf{(até 12/2018)} (mais um artigo sobre
  alinhamento e outro com a redução da complexidade para $O(n)$);
  \item \textbf{disponibilização do \textit{framework}}: finalização da codificação do
  \textit{framework} C++/Lua e documentação da API \textbf{(até 12/2018)};
  \item \textbf{defesa}: escrita da tese e defesa final \textbf{(até 12/2018)}.
\end{enumerate}
}
\end{document}
